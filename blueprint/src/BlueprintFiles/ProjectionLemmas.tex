\chapter{Projection Lemmas}

Let $\mathcal{A}$ be a unital $C^{*}$--algebra in this section. In this section we collect relevant results about (selfadjoint) projections in $\mathcal{A}$.
Recall that an element $p$ in a $C^{*}$--algebra is a projection if $p^2=p^{*}=p$. By the Spectral Mapping Theorem (or using the full Gelfand Duality),
a selfadjoint element $a\in A$ is a projection if and only if the spectrum of $a$ is contained in $\{0,1\}$.

\begin{lemma}
  \label{lem:proj_le_one}
  \lean{IsStarProjection.le_one}
  \mathlibok
  \uses{lem:selfadjoint_le_norm}
  For all projections $p\in \mathcal{A}$, $p \le 1$.
\end{lemma}

\begin{proof}
  \leanok
  We can get this by Gelfand duality (cf. Sakai 1.2.3) and Lemma \ref{lem:selfadjoint_le_norm}.
\end{proof}

\begin{lemma}
  \label{lem:proj_sub_pos_iff_comm_eq_self}
  \lean{IsStarProjection.le_tfae}
  \leanok
  \uses{lem:star_conj_pos, lem:proj_le_one}
  Let $\mathcal{A}$ be a $C^{*}$--algebra and $p,q \in \mathcal{A}$ be projections. Then $p-q\ge 0$ iff $qp = pq = q$.
\end{lemma}

\begin{proof}
  \leanok
  By Lemma \ref{lem:star_conj_pos}, if $p-q \ge 0$ then $q(p-q)q=qpq-q^3=qpq-q \ge 0$. By Lemma \ref{lem:proj_le_one}, we have $p\le 1$ and
  employing Lemma \ref{lem:star_conj_pos} we obtain $qpq\le q1q=q$. Since $qpq\ge q$ and $qpq\le q$ we have $qpq=q$, which implies that $q(p-q)q=0$.
  By the $C^{*}$-property of the norm, we have $\|(p-q)^{1/2}q\|^2=\|q(p-q)q\|=0$ and so $(p-q)^{1/2}q=0$ and therefore $(p-q)q=(p-q)^{1/2}(p-q)^{1/2}q=0$.
  It follows that $pq=q$, and taking adjoints, $qp=q$.
  Conversely, if $qp=pq=q$, one easily checks that $p-q$ is a projection and so its spectrum is contained in $\{0,1\}$ and it is positive.
\end{proof}

\begin{corollary}
  \label{cor:proj_sub_of_subproj}
  \lean{IsStarProjection.le_tfae}
  \leanok
  \uses{lem:proj_sub_pos_iff_comm_eq_self}
  For all $p,q \in \mathcal{A}$ projections such that $q\le p$, $p-q$ is a projection.
\end{corollary}
\begin{proof}
  \leanok
\end{proof}

The next results are specific to $W^{*}$--algebras.

\begin{proposition}
  \label{prop:one_sided_ideals_in_wstar_proj_Sak_1_10_1} (Sakai 1.10.1)
  \uses{lem:star_left_mul_right_mul_sig_cts_Sak_1_7_8,cor:sig_clos_of_cstar_is_wstar_Sak_1_7_9,lem:proj_sub_pos_iff_comm_eq_self,lem:wstar_unital}
  Let $\mathcal{L}$ (resp. $\mathcal{R}$) be a left (resp. right) $\sigma(M,M_{*})$--closed ideal of a $W^{*}$--algebra $M$.
  Then there exists a unique projection $p$ (resp. $q$) in $M$ so that $\mathcal{L}=Mp$ (resp. $\mathcal{R}=qM$).
\end{proposition}

\begin{proof}
  Suppose $\mathcal{L}$ is a $\sigma$--closed left ideal. By Lemma \ref{lem:star_left_mul_right_mul_sig_cts_Sak_1_7_8}, $\mathcal{L^{*}}$ is
  also $\sigma$--closed and $\mathcal{N}=\mathcal{L}\cap \mathcal{L}^{*}$ is a $\sigma$--closed $*$--subalgebra of $M$, and therefore by
  Corollary \ref{cor:sig_clos_of_cstar_is_wstar_Sak_1_7_9} is a $W^{*}$--subalgebra of $M$. Let $p$ be the identity of $\mathcal{N}$, which
  exists by Corollary \ref{lem:wstar_unital}. Since $\mathcal{L}$ is a left ideal, $xp\in \mathcal{L}$ for any $x\in M$, thus $Mp\subseteq \mathcal{L}$ conversely, $xp=x$for any
  $x\in \mathcal{L}$, since $p$ is the identity of $\mathcal{N}$, thus $\mathcal{L}\subseteq Mp$.
  If $p_1$ is another projection in $M$ such that $\mathcal{L}=Mp_1$ then $Mp=Mp_{1}$ and $p_1=xp$ for some $x\in M$. By the selfadjointness
  of $p_1$ we have $p_1=px^{*}xp$ and therefore $p_1p=pp_1=p_1$ and therefore $p_1\le p$ by Lemma \ref{lem:proj_sub_pos_iff_comm_eq_self}.
  We get $p\le p_1$ using the same argument. The claim for right ideals follows mutatis mutandis.
\end{proof}


\begin{proposition}
  \label{prop:proj_compl_lat_wstar_Sak_1_10_2} (Sakai 1.10.2)
  \uses{prop:one_sided_ideals_in_wstar_proj_Sak_1_10_1,lem:proj_sub_pos_iff_comm_eq_self}
  Let $M$ be a $W^{*}$--algebra. Its set $M^{p}$ of projections is a complete lattice with respect to $\le$.
\end{proposition}

\begin{proof}
  Let $\{e_{i}\}_{i\in \mathbb{I}}$ be any set of projections in $M$. Consider $\mathcal{L}_1$, the $\sigma$--closed left ideal
  generated by $\{Me_{i}\}_{i\in \mathbb{I}}$, and $\mathcal{L}_{2}=\bigcap_{i\in \mathbb{I}}Me_{i}$. By
  Proposition \ref{prop:one_sided_ideals_in_wstar_proj_Sak_1_10_1} there exist projections $e_1,e_2$ in $M$ so that
  $\mathcal{L}_1=Me_1$ and $\mathcal{L}_2=Me_{2}$. By Lemma \ref{lem:proj_sub_pos_iff_comm_eq_self}, we know $e_i\le e_1$
  and $e_2\le e_i$ for any $i\in \mathbb{I}$. (To prove the first of these, since $e_i=xe_1$ for some $x\in M$,
  we have by the self-adjointness of $e_i$ that $e_i=e_1x^{*}xe_1$ and so $e_1e_i=e_ie_1=e_i$ and so $e_i\le e_1$ by
  Lemma \ref{lem:proj_sub_pos_iff_comm_eq_self}. The other direction is analogous.)
  Furthermore, if $p\in M^{p}$ is another projection that is an upper bound for all the $e_i$ then $Me_1\subseteq Mp$ since
  $\mathcal{L}_1=Me_1$ is the left ideal generated by the $e_i$. It follows $e_1\le p$ by the same calculation we just did.
  If $p\in M^{p}$ is a lower bound for all the $e_i$ then $Mp\subseteq M_{e_i}$ for every $i\in \mathbb{I}$ and so
  $Mp\subseteq \bigcap_{i}Me_{i}=Me_{2}$ and we obtain $p\le e_2$. It follows that $e_1=\sup(e_i)$ and $e_2=\inf(e_i)$.
\end{proof}
